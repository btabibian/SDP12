\documentclass[12pt]{IEEEtran}

\usepackage[tight,footnotesize]{subfigure}

\ifCLASSINFOpdf
  \usepackage[pdftex]{graphicx}
  \graphicspath{{./images/}}
\else
\fi

\usepackage[cmex10]{amsmath}
\usepackage[a4paper]{ geometry}
\geometry{top=0.7in, bottom=0.7in, left=0.7in, right=0.7in}


\hyphenation{op-tical net-works semi-conduc-tor}

\begin{document}
	
\title{Individual Report 4}

\author{Calum Jackson (s0812597) - 
SDP Group 12}

\maketitle

%\numberwithin{equation}{section}
%\IEEEpeerreviewmaketitle
%\pagebreak

\section{Introduction}
This report, the fourth in the series, will document robot changes since Milestone 3, changes to the strategy code, and the robots performance in the friendly match and Milestone 4. The team had two main objectives to work towards:
\begin{itemize}
\item Ensure the robot was capable of competing in the upcoming friendly matches.
\item Implement a intercepting ball strategy for milestone 4.
\end{itemize}

\section{Strategy}
Focusing on the friendly match, the strategy code was improved in numerous areas.  The Strategy classes were integrated together using a MainStrategy class, which uses a state system to decide what strategy needs to be called to deal with the current on-pitch situation. Using the MainStrategy class allowed the strategy classes to be split up and made more specific, increasing their efficiency and accuracy. Functions was often moved to higher levels to increase their availability to all of the strategies, often specific functions were made more general to increase their usefulness, which led to the removal of similar functions.
Methods were created to overcome situations such as if a destination point was of the pitch, if the opponent has possession of the ball, or if the touch sensors were activated.
The method calculating the obstacle avoidance point were improved to implement a dynamic point instead of a static point. This was done by calculating the the point in relation to the robot, so the point would be updated as the robot moved.

\section{Robot Changes}
Touch sensors were added at the front of the robot, and implemented in the code, preventing the robot continually crashing into an obstacle. During the friendly matches it was apparent the plate holder was impeding access to the NXT brick, wasting game time when the robots were being reset. Instant access to the NXT brick was gained by putting the plate holder on a hinge.

\section{Testing}
Testing became a large part of the project, making sure our methods and strategies were working properly, noting scenario's our robot struggled with (such as if ball was close to the wall, which lead to a getBallFromWall strategy), and ensuring thresholds used in the simulator were appropriate on the pitch. This was time consuming, but useful in finding flaws and possibilities for improvements.

\section{Friendly Tournament}
The friendly tournament went well for our team, reaching the semi finals and losing only due to not scoring the first goal in a score draw. The simple strategies worked well in most situations, 

\section{Milestone 4}
We were unsuccessful in intercepting the ball in milestone 4, even though the robot was working well previous to the milestone. Our intercepting strategy could work well, managing to judge an intercept point, even if the ball was bounced off a wall. Unfortunately, last minute tweaks and implementation on untested ideas reduced our functionality significantly, and hopefully we will learn from this. 

\section{Future Improvements}
To maximise our performance in the final games, I feel we should focus on maximising the performance of the strategies currently being used, and integrating the PFS movement technique more effectively into the strategies. 

\section{Conclusion}
Our robot was not successful in the main objective for milestone 4, but after a positive performance in the friendly matches and progression of the PFS strategy, our robot is looking in a good position for the final match day. Many improvements and a lot of fine tuning is still needed to maximise our performance, but we currently have a good base to work with.


\end{document}
